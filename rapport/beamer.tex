\documentclass[10pt,xcolor={usenames,dvipsnames,svgnames,table}]{beamer}
\usepackage[utf8]{inputenc}
\usepackage[english]{babel}
\usepackage{lmodern}
\usetheme{Singapore}
\usepackage{subcaption}
\usepackage{fontenc}
\usepackage{graphicx}
\usepackage{amsfonts}
\usepackage{hyperref}
\usepackage{amssymb}
\usepackage[ruled,vlined,linesnumbered]{algorithm2e}
\usepackage{float}
\usepackage{amsthm}
\usepackage{amsmath}
\usepackage{amssymb}
\usepackage{mathrsfs}
\usepackage{mathtools}
\usepackage{tikz}
\definecolor{turquoise}{RGB}{255,127,0}
\usetikzlibrary{decorations.pathmorphing}
% \usetikzlibrary{arrows}
\usetikzlibrary{chains,fit,shapes}
\usetikzlibrary{arrows}
% \usepackage{fullpage}
\tikzset{
  treenode/.style = {align=center, inner sep=0pt, text centered,
    font=\sffamily},
  zigzag/.style = {->,black,very thick,line join=round,
    decorate, decoration={
      zigzag,
      segment length=4,
      amplitude=.9,post=lineto,
      post length=2pt
    }},
  zigzagred/.style = {->,red,very thick,line join=round,
    decorate, decoration={
      zigzag,
      segment length=4,
      amplitude=.9,post=lineto,
      post length=2pt
    }},
  arn_bb/.style = {treenode, circle, black, font=\sffamily\bfseries, draw=black,
    fill=white, text width=2.5em, very thick},% arbre rouge noir, noeud noir
  arn_ns/.style = {treenode, rectangle, white, font=\sffamily\bfseries, draw=black,
    fill=black, minimum width=1.5em, minimum height=1.5em, very thick},% arbre rouge noir, noeud noir
  arn_n/.style = {treenode, circle , white, font=\sffamily\bfseries, draw=black,
    fill=black, text width=1.5em},% arbre rouge noir, noeud noir
  arn_nb/.style = {treenode, circle , white, font=\sffamily\bfseries, draw=black,
    fill=black, text width=2.5em},
  arn_r/.style = {treenode, circle , red, draw=red, 
    text width=1.5em, very thick},% arbre rouge noir, noeud rouge
  arn_x/.style = {treenode, rectangle, draw=black,
    minimum width=1em, minimum height=1em},% arbre rouge noir, nil
  arn_b/.style = {treenode, circle , blue, draw=blue, 
    text width=1.5em, very thick},
  arn_g/.style = {treenode, circle , green, draw=green, 
    text width=1.5em, very thick},
  arn_y/.style = {treenode, circle , yellow, draw=yellow, 
    text width=1.5em, very thick},
  arn_pu/.style = {treenode, circle , purple, draw=purple, 
    text width=1.5em, very thick},
  arn_pi/.style = {treenode, circle , pink, draw=pink, 
    text width=1.5em, very thick},
  arn_t/.style = {treenode, circle , pink, draw=turquoise, 
    text width=1.5em, very thick},
  arn_rb/.style = {treenode, circle , red, draw=red, 
    text width=2.5em, very thick},
  arn_bs/.style = {treenode, rectangle , blue, draw=blue, 
    minimum width=1.5em, minimum height=1.5em, very thick},
  arn_rs/.style = {treenode, rectangle, red, draw=red, 
    minimum width=1.5em, minimum height=1.5em, very thick},% arbre rouge noir, noeud rouge
  arn_gs/.style = {treenode, rectangle , green, draw=green, 
    minimum width=1.5em, minimum height=1.5em, very thick},
  arn_ys/.style = {treenode, rectangle , yellow, draw=yellow, 
    minimum width=1.5em, minimum height=1.5em, very thick},
  arn_pus/.style = {treenode, rectangle , purple, draw=purple, 
    minimum width=1.5em, minimum height=1.5em, very thick},
  arn_pis/.style = {treenode, rectangle , pink, draw=pink, 
    minimum width=1.5em, minimum height=1.5em, very thick},
  arn_rbs/.style = {treenode, rectangle , red, draw=red, 
    minimum width=1.5em, minimum height=1.5em, very thick},
  arn_ts/.style = {treenode, rectangle, pink, draw=turquoise, 
    minimum width=1.5em, minimum height=1.5em, very thick}, % CMB
}
\author{Adrien HUSSON, Matthieu JOURNAULT, Lucas RANDAZZO}
\date{Vendredi 13 mars, 2015}
\title{Soutenance du projet Coq}
\newcommand{\eqname}[1]{\tag*{#1}}

\begin{document}
  \maketitle
  \begin{frame}
    \tableofcontents
  \end{frame}
  
  \section{Formalisation Système F stratifié en Coq}
  \begin{frame}
  \begin{block}{Partie 1}
  \begin{enumerate}
   \item Correction et complétude de l'inférence
   \item Regularity, Narrowing
   \item Stabilité du typage par substitution
   \item Cumulativité
   \item Red
  \end{enumerate}
  \end{block}
  \begin{alertblock}{Partie 2}
   Non fait
  \end{alertblock}
  \end{frame}
%  Summarize de ce qu'on a formulé, recall what the project was about, main step of the formalization, explain and comment formalization choices, describe ur personnal work, comment what was difficult/tedious, what would we do with more time.
%   Qu'est ce qui a été fait dans le projet
  \begin{frame}{Choix d'implémentation}
  \begin{block}{Indice de "de Bruijn"}
    \begin{itemize}
     \item indices de Bruijn
     \item liste commune de type \texttt{list envElem}
     \item $\texttt{envElem} = | \texttt{Typ} | \texttt{kind}$
    \end{itemize}

  \end{block}
  \begin{exampleblock}{Example}
   \[
    (\lambda 0 0)(\lambda 0 0) \rightarrow (\lambda 0 0)(\lambda 0 0) \rightarrow (\lambda 0 0)(\lambda 0 0) \rightarrow (\lambda 0 0)(\lambda 0 0) \rightarrow (\lambda 0 0)(\lambda 0 0) 
   \]
  \end{exampleblock}
  \end{frame}

  \begin{frame}{Etapes du développement}
  \begin{block}{Etapes du développement 1}
   Lemmes courants : préservation du typage et du kinding par {\tt insert\_kind}, {\tt remove\_var}, {\tt replace\_kind}, {\tt env\_subst}. 
  \end{block}
  \begin{block}{Etapes du développement 2}
   Symmétrie de {\tt insert\_kind} et {\tt remove\_var} 
  \end{block}
  \centering
  \begin{tabular}{c c c}
%     lol & lol & lol
   & Ajout & Enlève \\ \cline{2-3}
   Type & \multicolumn{1}{|c|}{ } & \multicolumn{1}{c|}{{\tt remove\_var}} \\ \cline{2-3}
   Kind & \multicolumn{1}{|c|}{{\tt insert\_kind}} & \multicolumn{1}{c|}{ } \\ \cline{2-3}
  \end{tabular}
  \end{frame}
  
  \begin{frame}{Choix d'implémentation (cont'd)}
  \begin{block}{Shift et Substitution}
  \begin{enumerate}
   \item {\tt tshift T f n} $= \uparrow_k^n T$
   \item Plusieurs lemmes généraux à prouver réutilisés dans toutes les preuves
  \end{enumerate}
  \end{block}
  \end{frame}
  
  \begin{frame}{Shift et Substitution}
  \begin{align} 
  \Big\uparrow_p^0 T &= T \eqname{\texttt{(tshift\_ident)}} \\
  \Big\uparrow_{p+d}^c \Big\uparrow_p^a T &= \Big\uparrow_{p}^a \Big\uparrow_{(d-a+p)}^c T 
       \eqname{\texttt{(tshift\_commut)}} \\
      \Big\uparrow_{n}^{y+x+k} T &= \Big\uparrow_{n+y}^k \Big\uparrow_{n}^{y+x} T
       \eqname{\texttt{(tshift\_merge)}} \\
      \Big\uparrow_{n+m}^{1} T[n \rightarrow U] 
       &= (\Big\uparrow^{1}_{n+m+1} T)[n \rightarrow \Big\uparrow^{1}_{n+m} U]
       \eqname{\texttt{(tsubst\_tshift\_swapDown)}}\\
  (\Big\uparrow^k_x T) [x+k+n \rightarrow \Big\uparrow^{x+r+k}_0 U] 
  &= \Big\uparrow^k_x T[x+n \rightarrow \Big\uparrow^{x+r}_0 U]
       \eqname{\texttt{(tsubst\_tshift\_swapUp)}} \\
     (\Big\uparrow^{k+1}_x T)[x+k \rightarrow U] &= \Big\uparrow^k_x T
       \eqname{\texttt{(tsubst\_tshift\_swapIn)}} \\
      T[k \rightarrow \Big\uparrow^k_0 U][k+x \rightarrow \Big\uparrow^k_0 V] &=
       T[k+x+1 \rightarrow \Big\uparrow^{k+1}_0 V][k \rightarrow \Big\uparrow^k_0 U[x \rightarrow V]]
       \eqname{\texttt{(tsubst\_commut\_general)}} 
  \end{align}
  \end{frame}

  \begin{frame}{Difficultés et améliorations possibles}
   \begin{block}{Lemmes de substitution/shifting}
    \begin{enumerate}
     \item Beaucoup de temps pour trouver les bons lemmes
     \item But : trouver des lemmes plus généraux dont les précédents se déduisent
     \end{enumerate}
   \end{block}

  \end{frame}
  
  \begin{frame}{Dépendances de lemmes}
  Swag graphs
  \end{frame}
  
\end{document}
